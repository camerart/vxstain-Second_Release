\documentclass[compsoc]{IEEEtran}

\usepackage{amsmath,amssymb}
\usepackage{caption,subcaption}
\usepackage[colorlinks]{hyperref}

\title{Introduction}
\author{FirstA LastA \and FirstB LastB\thanks{Funding and author information}}
% \markboth{}{}

\begin{document}

\IEEEtitleabstractindextext{
\begin{abstract} This is the abstract.
  
\end{abstract}

\begin{IEEEkeywords}
  
\end{IEEEkeywords}}

\maketitle

\section{Introduction}
\label{sec:introduction}

\IEEEPARstart{A}{ccording} to~\cite{ham_deep_2019}, multi-year ENSO can be predicted
using deep convolutional networks.

\section{Related Works}
\label{sec:related-works}

\section{SlowFast Cloud Tracking}
\label{sec:slowf-cloud-track}

Example equation:
\begin{equation}
  \label{eq:example}
  a^{2} + b^{2} = c^{2}.
\end{equation}

As given by~\eqref{eq:example}, the blabla. 

\section{Experiments}
\label{sec:experiments}

\subsection{Ablation Study}
\label{sec:ablation-study}

Ablation study is shown in Table~\ref{tab:ablation}.

\begin{table}[h]
  \centering
  \caption{Ablation Study}
  \label{tab:ablation}
  % Generally, tables are with only three horizontal lines
  \begin{tabular}{ccc|ccc}
    \hline
    Component A & Comp B & Comp C & Metric A & M-B & M-C\\
    \hline
    & & & & & \\
    & & & & & \\
    \hline
  \end{tabular}
\end{table}

\subsection{Comparison with SOTA}
\label{sec:comparison-with-sota}

\begin{figure}[h]
  \centering
  % To show figures, replace \makebox with \includegraphics
  % To show a whole figure, using only \includegraphics
  % To show a figure occupies both columns, using environment figure*
  \subfloat[Sub-caption]{\makebox[0.5\linewidth]{Sub Figure A}}
  \subfloat[Sub-caption]{\makebox[0.5\linewidth]{Sub Figure B}}
  \caption{Results}
  \label{fig:results}
\end{figure}

As shown in Fig.~\ref{fig:results}.

\section{Conclusions}
\label{sec:conclusions}


\bibliographystyle{IEEEtran}
\bibliography{IEEEabrv,refs}

\end{document}

%%% Local Variables:
%%% TeX-engine: default
%%% End: